\documentclass[a4paper]{book}

\usepackage[brazil]{babel}
\usepackage{cmap}
\usepackage{lmodern}
\usepackage{lastpage}
\usepackage{indentfirst}
\usepackage{color}
\usepackage[utf8]{inputenc}
\usepackage[T1]{fontenc}
\usepackage{hyperref}
\usepackage{graphicx}
\usepackage[brazilian,hyperpageref]{backref}
\usepackage{enumitem}
\usepackage{footnote}
\hypersetup{
    colorlinks=false,
    pdfborder={0 0 0},
}
\setlist{noitemsep}

\bibliographystyle{ieeetr}

\title{Fraseologia Musical}
\author{Esther Scliar}

\begin{document}

\maketitle

\chapter*{A obra inédita de Esther Scliar}

Poucos são os músicos, em todo país, que não conhecem Esther Scliar. Durante anos, foi professora de todos os seminários de música importantes realizados entre nós. Mas são igualmente poucos os que conhecem a extensa obra que deixou, praticamente inédita.

Deve-se a Leonor Scliar Cabral o surgimento deste primeiro volume, intitulado \textit{Fraseologia Musical}. Devido à sua dedicação, toda a obra foi reunida, estudada. Leonor Scliar teve o cuidado de reunir musicólogos que pudessem avaliar a importância e extensão do trabalho deixado por sua irmã.

Ainda estão sendo coletadas algumas composições de música para coro que desapareceram em meio ao intensivo trabalho de seminários e conferências.

Já se pode ter, agora, pelos volumes organizados, uma noção da importância da obra de Esther Scliar. Na medida em que os próximos volumes aparecerem, tanto os ensaios quanto as composições, o público descobrirá uma das maiores expressões da música brasileira.

Os alunos de Esther Scliar dos seminários de música ou seus cantores dos inúmeros corais que ela organizou, pelo país afora, talvez não se espantem com a sua vasta obra, porque de alguma forma já a conheciam mas a maioria haverá de se impressionar com aquilo que ela nos legou.

Na década de 1950, em Porto Alegre, Esther Scliar ainda era regente de um coral cujos componentes, pouco depois, fundaram o Coral da Filosofia, orientado por Madeleine Ruffier. Ela  preferia compor e ouvir suas músicas executadas e discuti-las com os regentes e os componentes do grupo. Várias vezes participamos deste trabalho; foi assim que a conhecemos.

Aos poucos, nos seminários, como nos de Teresópolis, que reunia os principais músicos do país, Esther Scliar foi firmando seu nome no campo da teoria, onde ela alcançaria sua plenitude, de acordo com o depoimento de seus alunos.

Seu desaparecimento súbito nos apanhou desprevenidos. Os que foram participantes de seus corais ou de suas aulas de regência e composição ainda guardam o calor humano com que Esther impregnava tudo o que fazia. Agora um novo público, com a gradativa publicação de sua obra inédita, poderá comprovar o talento de uma das maiores personalidades da música brasileira contemporânea.

\begin{flushright}
Carlos Jorge Appel
\end{flushright}

\chapter*{Prefácio}

Numa linguagem concisa, sem o acúmulo de palavras a cuja presença em torno da arte Valéry aludia criticamente, neste livro, Esther Scliar, decantando longos anos de fecunda experiência didática, mergulha numa temática abstrata e pouco trilhada, desentranhando exaustivamente alguns dos mecanismos que ligam os elementos no discurso musical.

Compositora, musicóloga, intérprete, professora exemplar, ela soube encaminhar gerações de artistas para um estudo sério, crítico das diversas fases de música, especialmente na área da música contemporânea.

A bibliografia musical brasileira padece de uma falta crônica de material escrito no campo da musicologia, da pedagogia ou da estética musical. As contadas exceções de trabalhos universitários ou pesquisas de fôlego devem-se ao esforço pessoal, poucas vezes reconhecido, de alguns destemidos estudiosos que, com enorme dificuldade, conseguem às vezes vencer a indiferença do meio ou a letargia das casas editoras. Uma segunda e penosa fase começará então nas prateleiras das livrarias para chegar às mãos de algum estudante que por própria iniciativa se disponha a lê-lo\ldots

Neste contexto, fazendo parte de um ambicioso projeto da autora, o livro integra-se também numa visão ampla onde o fenômeno musical é analisado sob diferentes pontos de vista.

A autora pesquisou numerosos autores e obras, trazendo para este livro grande quantidade de usos e modos, em forma de exemplos do texto, convertendo o livro num dos mais documentados sobre o assunto.

Em suma, um livro capaz de introduzir ao estudo e para ser usado como material de base em mãos de estudiosos e amantes da música.

\begin{flushright}
Conrado Silva
\end{flushright}

\chapter*{Fraseologia}

Em sua projeção temporal os sons tendem a se articular em pequenos agrupamentos delimitados por cesuras. Estes agrupamentos concatenam-se entre si, formando conjuntos maiores, os quais se encadeiam com os seguintes, formando novos grupos. O caráter desta projeção é sintático, semelhante ao discurso verbal. Seu estudo: Fraseologia.

Agrupamento fraseológico ``é o produto da semelhança, diferença, proximidade ou separação dos sons percebidos pelo ouvido e organizados pela mente.''\cite{cooper1960rhythmic}

\chapter{Da semelhança e diferença}

Quanto maior a afinidade entre os componentes estruturais, tanto maior a coesão e vice-versa.

\section{Fatores básicos de coesão}

\noindent
\begin{savenotes}
\begin{center}
\begin{tabular}{|p{6.5cm}|p{4.7cm}|}
\hline

\begin{center}
\textbf{Alturas} 
\end{center}

& 

\begin{center}
\textbf{Durações} 
\end{center}

\\ 

\hline

\begin{itemize}[leftmargin=0.3cm]
\item Movimento unidirecional
\item Homogeneidade do gênero 
\item Homogeneidade ou coerência harmônica
\item Homogeneidade de configuração\footnote{Esclarecimento posterior.}
\item Textura homogênea
\end{itemize}

&

\begin{itemize}[leftmargin=0.3cm]
\item Emissões isócronas 
\item Configurações homogêneas\footnote{idem.}
\end{itemize}

\\ 
\hline
\end{tabular} 
\end{center}
\end{savenotes}

N. B. --- Embora a homogeneidade dos timbres condicione a coesão, torna-se inoperante sua especificação na estrutura da sintaxe.

\chapter{Da proximidade e separação}

\section*{Alturas}

\begin{center}
\begin{savenotes}
\begin{tabular}{lcccr}
\textbf{Proximidade} & & & & \textbf{Separação} \\
Idênticos ou contíguos & $\leftarrow$ & Registro(s) & $\rightarrow$ & Distanciados \\
Mínimo (semitom)\footnote{No temperamento igual.} & $\leftarrow$ & Espaço melódico & $\rightarrow$ & Máximo \\
Notas atrativas & $\leftarrow$ & Espaço harmônico & $\rightarrow$ & \begin{tabular}{@{}r@{}} Resolução (espe- \\ cialmente a \\ cadência)\end{tabular} \\
\end{tabular}
\end{savenotes}
\end{center}

\section*{Durações}

\subsubsection*{Pausa}

Quanto mais longa, tanto maior a separação.

\subsubsection*{Velocidade das emissões}

Quanto mais rápida, tanto maior o nexo com o próximo som e vice-versa.

\paragraph{Observação:} Contraditoriamente, a semelhança de configurações melódicas ou rítmicas condiciona a coesão e a separação dos agrupamentos. Em outras palavras: a homogeneidade propicia a coesão do conjunto; os agrupamentos, porém, são autônomos, estando separados por cesuras.

\begin{figure}
\caption{Exemplo 1}
\end{figure}

Os elementos fraseológicos subordinam-se às leis do movimento. A fraseologia é, po¡s, uma manifestação do \textit{ritmo}.

Como foi visto anteriormente, cada movimento é constituído de duas fases:

\begin{enumerate}[label=\emph{\alph*})]
\item Impulso ou \textit{arsis}
\item Apoio ou \textit{thesis}
\end{enumerate}

Em muitos agrupamentos, a \textit{arsis} está omitida. Estando presente, constituí a fase inicial do movimento, cabendo à \textit{thesis} a complementação. Cada um dos movimentos possui uma tendência específica. O impulso é a manifestação de energia, tendendo a durar o menos possível enquanto que o apoio se identifica com o repouso, tendendo a durar mais.

Cada agrupamento musical pode ser classificado de duas maneiras:
\begin{itemize}
\item quanto ao início
\item quanto ao final
\end{itemize}

Em cada uma das maneiras, há várias possibilidades:

\subsection*{Quanto ao início}
\begin{itemize}
\item Se o agrupamento iniciar com a \textit{thesis}, o ritmo é \emph{tético}. Exemplos 2a; 2b
\item Se o agrupamento iniciar com a \textit{arsis}, o ritmo é \emph{anacrúsico}\footnote{O fragmento que precede a \textit{thesis}, denomina-se anacruse}. Exemplos 3a; 3b.
\item Se o agrupamento iniciar após a \textit{thesis}, com precedência de pausa não superior a um tempo, o ritmo é \emph{acéfalo}, \emph{decapitado} ou \emph{procatalético}. Exemplo 4.
\end{itemize}

\paragraph*{Observação}

\begin{itemize}
\item Algumas vezes, o desenho se inicia na Thesis, mas seu caráter é anacrúsico. Este caráter se deve fundamentalmente ao parâmetro duração, ou seja, rapidez das emissões, muitas vezes reforçadas por saltos e movimento ascendente. Exemplo 5.
\item Sob o ponto de vista perceptivo, o ritmo \emph{acéfalo} se relaciona diretamente com o \emph{tético}, pela supressão de seu início. Entretanto, este relacionamento só se evidencia quando:
\begin{enumerate}
\item houver reprodução de agrupamento -- lei da simetria. Exemplo 6a.
\item a duração dos agrupamentos for relativamente longa. Exemplo 6b.
\end{enumerate}
O segundo aspecto é mais importante, pois os sons de breve duração tendem a se concatenar com os seguintes, caracterizando por seu ritmo impulsivo, os ritmos anacrúsicos. Exemplo 6c.
\end{itemize} 

\subsection*{Quanto ao final}
\begin{itemize}
\item \textbf{Terminação Masculina:} o último som do agrupamento coincide com o apoio. Exemplo 7a; 7b.
\item \textbf{Terminação Feminina:} o último som do agrupamento é posterior ao apoio. Exemplos 8a; 8b.
\item \textbf{Terminação Feminina com caráter Masculino:} finaliza após a Thesis, sendo, porém, acentuada., em virtude dos fatores condicioe nantes do acento: precedência de va|ores menores e salto,_ por exemp|0. Exemplo 9.
\end{itemize}

Exemplo 9
J. B. Loe¡Ilet, Suite em So| rn (Sarabanda)

tr

    

T r jr

Termínação Masculina com caráter Feminino: coincide com a
Thesis, todavía, seu efeíto se equipara à_ termínação femínina; reso-

16


\bibliography{fraseologia}

\end{document}